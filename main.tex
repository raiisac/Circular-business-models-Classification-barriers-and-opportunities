\documentclass[a4paper,man,10pt]{apa6}
\usepackage[utf8]{inputenc}
\usepackage[english]{babel}
%\usepackage{natbib}
%\setcitestyle{square, numbers}
%\bibliographystyle{ksfh_nat}

%\documentclass{article}

\usepackage[square,numbers, sort&compress]{natbib}
\bibliographystyle{ieeetr}
%\usepackage[
%    backend=biber,
%    style=chem-angew,
%  ]{biblatex}
 
% \addbibresource{example}
\usepackage{amsmath}
\usepackage{graphicx}
\usepackage[colorinlistoftodos]{todonotes}

\title{Circular business models: Classification, barriers, and opportunities}
\shorttitle{Circular business models}
\author{Ra\"{i}sa Carmen}
\affiliation{KU Leuven, Campus Brussels}

\abstract{This report starts with an overview of different types of circular business models. Many classifications already exist in academic literature and this report will aim to deliver a general, integrated overview of many of them. Each business model has specific features that might make it either exceptionally fitting or inappropriate for some markets (but not others) for transitioning towards a circular economy. We outline opportunities and threats for each of the business models and add policy recommendations to lift barriers and leverage the power of these business models for a transition towards a circular economy. The theoretical and conceptual overview is supplemented with practical implementations and examples, mainly from Belgium. }
\setcounter{secnumdepth}{3}

\begin{document}
\renewcommand{\baselinestretch}{1}
\maketitle

\section{Introduction}\label{sec:Introduction}
According to the Belgian government \citep{Belgium.be2018}, ``Circular economy is an economic and industrial system which aims to keep products, their components and materials in circulation as long as possible within the system, while ensuring the quality of their use. As such, the circular economy contrasts with the linear economy, in which products and materials are disposed of at the end of their economic life.' To achieve this will require great effort from industry to rethink how they do business. A company's \emph{business model} is ``the rationale of how an organization
creates, delivers and captures value'' \citep[p.14]{osterwalder2010business}. It describes the key activities or components are interconnected within the organization and throughout the supply chain. Through these interconnections, the organization will aim to generate value or profit. The business model canvas (BMC) breaks this down into nine components; customer value proposition, segments, and relationships, channels, key resources, key
activities, partners, costs, and revenues. This should help organizations in generating, improving, and aligning the components of a successful BM. One downside of the BMC, though, is that there is little room for social or environmental considerations, economic profit is the main goal. To overcome this shortcoming, 
\cite{JOYCE20161474} extended the BMC to a triple layered business model canvas (TLBMC). In line with the 'triple bottom line' perspective that aims to satisfy people, planet, and profit \citep{Elkington1998}, the TLBMC requires businesses to concretize their financial, social, and environmental goals in three canvases. Each canvas needs to be internally consistent but coherence across all three is also essential.

Now that policy makers are pushing industry to transition to a circular economy, the triple bottom line principles, efficient resource use, and 


\section{Circular business models}
\label{sec:CBS}
Many classifications and frameworks are already existing in the academic literature. In section~\ref{subsec:literature}, we give a non-exhaustive overview of some well-established frameworks. Next, section~\ref{subsec:focus} defines which CBMs the remainder of this paper will focus on each BM's characteristics, how we distinguish them, and why they were selected. 

\subsection{Literature review}\label{subsec:literature}


\subsubsection{Product-service systems}\label{subsubsec:PSS}
A special volume of the Journal of Cleaner Production was dedicated research on why sustainable PSS had \textbf{not} been implemented.


\subsubsection{Sharing economy}\label{subsubsec:sharing}
The sharing economy is often seen as a great driver for a circular economy but the term is clouded by unclear and overlapping concepts such as peer economy, gig economy, access economy, and collaborative consumption. 
In a clarifying article, Rachel Botsman (author of the influential book ``What's Mine is Yours: How Collaborative Consumption Is Changing The Way We Live''\citep{book:815976}) distinguishes four systems that are commonly classified under the term `sharing economy'. While all systems aim to match what one person has with another person's wants to unlock the value of underused assets, result in a more distributed power, and often rely on new digital and communication technologies for trust and efficiency, they differ substantially on other aspects. We list the four systems below as specified in \citep{Botsman2015}.

\begin{description}
\item[Collaborative economy] An economic system of decentralized networks and marketplaces that unlocks the value of underused assets by matching needs and haves, in ways that bypass traditional middlemen, such as Etsy and Kickstarter.
\item[Sharing economy] An economic system based on sharing underused assets or services, for free or for a fee, directly from individuals such as Airbnb and BlaBlaCar.
\item[Collaborative consumption] The reinvention of traditional market behaviors—renting, lending, swapping, sharing, bartering, gifting—through technology, taking place in ways and on a scale not possible before the internet such as Zipcar, Freecycle and eBay.
\item[On-demand economy] Platforms that directly match customer needs with providers to immediately deliver goods and services such as Uber.
\end{description}


\begin{table} \small \centering
\begin{tabular}{p{0.3 \linewidth}|p{0.7 \linewidth}}
Reference & BM categorization\\ \hline
     \cite{Lacy2015, Menkveld2018}& (1) circular supply chain, (2) recovery \& recycling, (3) product life-extension, (4) sharing platform, (5) product as a service \\
      &\\
     
     
\end{tabular}\caption{Overview of circular business model categorizations}\label{tab:BMcat}
\end{table}


\todo[inline, color=green!40]{An overview of existing classifications}
\subsection{Focus and contribution}\label{subsec:focus}
\todo[inline, color=green!40]{describe here which CBMs we'll focus on in this paper and why.}


\section{Opportunities and enablers}\label{sec:opportunities}
\subsection{Decrease costs} \label{subsec:DecreaseCosts}
Overcome rising price of scarce resources

Reduces costs of compliance and waste management

\subsection{Increase revenues}\label{subsec:IncreaseRevenues} 
Increased revenues by selling waste

New revenues from second life

Use interaction with customer for disposal as a selling opportunity

\subsection{Expand knowledge base}\label{subsec:Knowledge}
Learn from wear and tear on disposed products to improve product design

\subsection{Build long-term relationships}\label{subsec:relationships}
Use interaction with customer for disposal as a selling opportunity

Customers like greater convenience,lower prices, better product/service

Greater customer loyalty


\begin{table} \tiny \centering
\begin{tabular}{p{0.2\linewidth}|p{0.1\linewidth} p{0.1\linewidth} p{0.1\linewidth} p{0.1\linewidth} p{0.1 \linewidth}}
Opportunity & circular supply chain &recovery \& recycling& product life-extension& sharing platform& product as a service \\\hline
overcome rising price of scarce resources & \cite{Lacy2015} &\cite{Lacy2015} & & &\\
Increased revenues by selling waste & &\cite{Lacy2015}  & & &\\
Reduces costs of compliance and waste management& &\cite{Lacy2015}  & & &\\
Use interaction with customer for disposal as a selling opportunity& &\cite{Lacy2015}  & & &\\
Learn from wear and tear on disposed products to improve product design& &\cite{Lacy2015}  & & &\\
New revenues from second life&& &\cite{Lacy2015}  & &\\
Customers like greater convenience,lower prices, better product/service&& &&\cite{Lacy2015}  &\\
Greater customer loyalty&& &&&\cite{Lacy2015}  \\

\hline
\end{tabular}\caption{Opportunities and enablers for circular business models as specified in literature}\label{tab:opportunities}
\end{table}

\section{Barriers, threats and challenges}\label{sec:barriers}

\subsection{Consumer reluctance}\label{subsec:Reluctance}
-they doubt the quality of refurbished/recycled goods. The market for secondary raw materials is underdeveloped, possibly due to high demand and supply uncertainties

-B2B sharing is underdeveloped

-fraud and trust concerns 

-Higher up-front costs may deter customers

Moral hazard

-Paas often needs higher population density to be successfull \citep{Menkveld2018}

\subsection{Financial obstacles}\label{subsec:financial}

Long, expensive R\&D 

Only for products with a longer lifespan (product life-extention)

Long pay-back times for initial investment
Products need to be more durable and might need additional expensive sensors or trackers in PaaS BMs \citep{Menkveld2018}

\subsection{Operational/technical obstacles}

Finding affordable returning and reprocessing strategies

geographically dispersed services

Bad product design may make recycling difficult

Mostly high-value equipment in B2B environment

\subsection{regulatory obstacles}\label{subsec:regulation}
Lagging regulatory framework

Exploitation and poor working conditions

\subsection{Marketing challenges}\label{subsec:marketing}
Companies are challenges to bind customers for a longer time for upgrades, add-ons, take-back 

\subsection{Overview}
\begin{table} \tiny \centering
\begin{tabular}{p{0.2\linewidth}|p{0.12\linewidth} p{0.12\linewidth} p{0.12\linewidth} p{0.12\linewidth} p{0.12 \linewidth}}
Barrier / threat & circular supply chain &recovery \& recycling& product life-extension& sharing platform& product as a service \\\hline
consumers are reluctant & \cite{Lacy2015} & & & \cite{Lacy2015}&\\
Long, expensive R\&D & \cite{Lacy2015} & & & &\\
Finding affordable returning and reprocessing strategies& & \cite{Lacy2015} & & &\\
Bad product design may make recycling difficult  & & \cite{Lacy2015} & & &\\
Only for products with a longer lifespan&& &\cite{Lacy2015}  & &\\
Mostly high-value equipment in B2B environment&& &\cite{Lacy2015}  & &\\
Higher up-front costs may deter customers&& &\cite{Lacy2015}  & &\\
Companies are challenges to bind customers for a longer time for upgrades, add-ons, take-back && &\cite{Lacy2015}  & &\\
B2B sharing is underdeveloped &&&&\cite{Lacy2015}&\\
fraud and trust concerns&&&&\cite{Lacy2015}&\\
Exploitation and poor working conditions&&&&\cite{Lacy2015}&\\
geographically dispersed services&& &&&\cite{Lacy2015}  \\
Long pay-back times for initial investment&& &&&\cite{Lacy2015}  \\
Moral hazard&& &&&\cite{Lacy2015}  \\
Lagging regulatory framework&& &&&\cite{Lacy2015}  \\
\hline
\end{tabular}\caption{Barriers, threaths and challenges as specified in literature}\label{tab:barriers}
\end{table}



%\section{References}
%\bibliographystyle{numeric}
\bibliography{example}
\appendix



% Commands to include a figure:
\begin{figure}
\centering
%\includegraphics[width=0.5\textwidth]{frog.jpg}
\caption{\label{fig:frog}This is a figure caption.}
\end{figure}

\end{document}

%
% Please see the package documentation for more information
% on the APA6 document class:
%
% http://www.ctan.org/pkg/apa6
%